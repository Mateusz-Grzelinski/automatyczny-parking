\chapter{SRS - specyfikacja wymagań}

\section{Ogólny diagram przypadków użycia}

\begin{figure}[H]
	\centering
	\includegraphics[width=100mm]{diagramy/PrzypUzycia.pdf}
	\caption{Przypadki użycia }
\end{figure}

\section{Definicje przypadków użycia}
\subsection{Obsługa terminala (identyfikator: UC3)}
\textbf{Aktorzy: }Klient, System przetwarzający obraz, Operator

\hspace{0cm}\textbf{Zakres: }System automatycznego parkingu

\hspace{0cm}\textbf{Poziom: }Systemowy

\hspace{0cm}\textbf{Udziałowcy i ich cele: }Klient chce dokonać zapłaty za postój, system musi pobrać wszystkie niezbędne dane oraz pobrać płatność

\hspace{0cm}\textbf{Zdarzenie wyzwalające (trigger): }Klienta wybiera funkcję Zapłać za postój

\hspace{0cm}\textbf{Warunki wstępne: }Tablica rejestracyjna pojazdu klienta musi znajdować się w systemie i być poprawnie odczytana

\hspace{0cm}\textbf{Warunki końcowe dla sukcesu: }
Tablica rejestracyjna zostaje potwierdzona przez klienta, płatność zostaje potwierdzona przez system

\hspace{0cm}\textbf{Warunki końcowe dla niepowodzenia: }Tablica rejestracyjna nie zostaje potwierdzona, płatność nie zostaje przetworzona \newline

\hspace{0cm}\textbf{Scenariusz główny: }
\begin{enumerate}
\item System wyświetla formularz wprowadzania rejestracji
\item Klient wprowadza rejestrację do systemu
\item System weryfikuje wprowadzoną rejestrację
\item System wyświetla odpowiednią rejestrację
\item Klient potwierdza zdjęcie rejestracji
\item System wyświetla kwotę do zapłaty
\item Klient dokonuje zapłaty za postój
\item System przetwarza płatność (include UC2)
\item System wyświetla potwierdzenie zapłaty
\end{enumerate}
\hspace{0cm}\textbf{Scenariusz alternatywny: }
\begin{enumerate}
\item[3.a] Wprowadzona rejestracja nie zostaje znaleziona
\item[3.a.1] Błąd zostaje zgłoszony operatorowi
\item[3.a.2] Po poprawieniu przez operatora danych następuje powrót do punktu 3 scenariusza głównego 
\end{enumerate}
\hspace{0cm}\textbf{Scenariusz alternatywny: }
\begin{enumerate}
\item[5.a] Klient zgłasza niezgodność wprowadzonej rejestracji i wyświetlonego zdjęcia operatorowi
\item[5.a.1] Po poprawieniu przez operatora danych następuje powrót do punktu 3 scenariusza głównego.
\end{enumerate}
\hspace{0cm}\textbf{Scenariusz alternatywny: }
\begin{enumerate}
\item[5.a] Klient anuluje wpisaną przez siebie rejestrację 
\item[5.a.1] Następuje powrót do punktu 1 scenariusza głównego
\end{enumerate}
\subsection{Wjazd na parking (identyfikator: UC1)}
\textbf{Aktorzy: }System przetwarzający obraz, Szlaban, Operator

\hspace{0cm}\textbf{Zakres: } System automatycznego parkingu

\hspace{0cm}\textbf{Poziom: } Systemowy

\hspace{0cm}\textbf{Udziałowcy i ich cele: } Klient chce wjechać na parking, System przetwarzający obraz musi rozpoznać tablicę rejestracyjną

\hspace{0cm}\textbf{Zdarzenie wyzwalające (trigger): } Klient podjeżdza pojazdem pod szlaban

\hspace{0cm}\textbf{Warunki wstępne: } Tablica rejestracyjna pojazdu jest widoczna i możliwa do odczytania przez system przetwarzający obraz

\hspace{0cm}\textbf{Warunki końcowe dla sukcesu: }
System przetwarzający obraz poprawnie odczytuje tablicę rejestracyjną, szlaban otwiera się

\hspace{0cm}\textbf{Warunki końcowe dla niepowodzenia: } Szlaban nie otwiera się \newline

\hspace{0cm}\textbf{Scenariusz główny: }
\begin{enumerate}
\item System wykrywa podjazd pojazdu pod szlaban
\item System przetwarzający obraz robi zdjęcie
\item System przetwarzający obraz odczytuje tablicę rejestracyjną
\item System automatycznego parkingu zapisuje datę i czas wjazdu pojazdu
\item Szlaban otwiera się (include: UC4)
\item Klient wjeżdża na parking
\item Szlaban zamyka się (include: UC4)
\end{enumerate}
\hspace{0cm}\textbf{Scenariusz alternatywny: }
\begin{enumerate}
\item[3.a] System przetwarzający obraz nie może odczytać tablicy rejestracyjnej
\item[3.a.1] Operator zostaje powiadomiony o błędzie.
\item[3.a.2] Operator wpisuje rejestracje do systemu.
\item[3.a.3] Następuje powrót do punktu 4 scenariusza głównego.
\end{enumerate}


\subsection{Wyjazd z parkingu (identyfikator: UC5)}
\textbf{Aktorzy: }System przetwarzający obraz, Szlaban, Operator

\hspace{0cm}\textbf{Zakres: }System automatycznego parkingu

\hspace{0cm}\textbf{Poziom: }Systemowy

\hspace{0cm}\textbf{Udziałowcy i ich cele: }Klient chce wyjechać z parkingu

\hspace{0cm}\textbf{Zdarzenie wyzwalające (trigger): } System odlicza 15 minut na wyjazd z parkingu

\hspace{0cm}\textbf{Warunki wstępne: }
Rejestracja pojazdu została poprawnie odczytana przy wjeździe

\hspace{0cm}\textbf{Warunki końcowe dla sukcesu: }Klient opuścił parking

\hspace{0cm}\textbf{Warunki końcowe dla niepowodzenia: }Płatność nie została potwierdzona, Czas na wyjazd z parkingu skończył się \newline

\hspace{0cm}\textbf{Scenariusz główny: }
\begin{enumerate}
\item System wykrywa podjazd podjazdu pod szlaban
\item System przetwarzający obraz ponownie robi zdjęcie
\item System przetwarzający obraz rozpoznaje tablicę rejestracyjną
\item System automatycznego parkingu weryfikuje dane: numer rejestracyjny, potwierdzenie płatności, pozostały czas wyjazdu
\item Szlaban otwiera się (include: UC4)
\item Klient opuszcza parking
\item Szlaban zamyka się (include: UC4)
\end{enumerate}
\hspace{0cm}\textbf{Scenariusz alternatywny: }
\begin{enumerate}
\item[3.a] System przetwarzający obraz nie może rozpoznać tablicy rejestracyjnej
\item[3.a.1] Operator zostaje powiadomiony o błędzie
\item[3.a.2] Operator poprawia dane w systemie
\item[3.a.3] Następuje powrót do punktu 4 scenariusza głównego.
\end{enumerate}

\subsection{Prezentacja statystyk (identyfikator: UC8)}
\textbf{Aktorzy: }Właściciel

\hspace{0cm}\textbf{Zakres: }System automatycznego parkingu

\hspace{0cm}\textbf{Poziom: }systemowy

\hspace{0cm}\textbf{Udziałowcy i ich cele: } Właściciel chce wyświetlić statystyki

\hspace{0cm}\textbf{Zdarzenie wyzwalające (trigger): } Właściciel wywołuje funkcję Pokaż statystyki

\hspace{0cm}\textbf{Warunki wstępne: }
Właściciel musi być zalogowany

\hspace{0cm}\textbf{Warunki końcowe dla sukcesu: } System prezentuje wybrane statystyki

\hspace{0cm}\textbf{Warunki końcowe dla niepowodzenia: } System nie wyświetla wybranych statystyk \newline

\hspace{0cm}\textbf{Scenariusz główny: }
\begin{enumerate}
\item System wyświetla zakres (tygodniowe, miesięczne, roczne) i typ statystyk (czas pobytu na parkingu, ilość korzystających z parkingu w zależności od pory dnia)
\item Właściciel wybiera odpowiedni zakres i typ statystyk
\item System oblicza wybrane statystyki
\item System prezentuje wyniki
\end{enumerate}

\subsection{Poprawa danych w systemie (identyfikator: UC7)}
\textbf{Aktorzy: }Operator

\hspace{0cm}\textbf{Zakres: }System automatycznego parkingu

\hspace{0cm}\textbf{Poziom: }Systemowy

\hspace{0cm}\textbf{Udziałowcy i ich cele: }Operator chce poprawić błędne dane w systemie

\hspace{0cm}\textbf{Zdarzenie wyzwalające (trigger): }Operator wywołuje funkcję Popraw dane

\hspace{0cm}\textbf{Warunki wstępne: }
Operator musi byc zalogowany

\hspace{0cm}\textbf{Warunki końcowe dla sukcesu: }Do systemu zostają wprowadzone poprawne dane

\hspace{0cm}\textbf{Warunki końcowe dla niepowodzenia: }Brak zdjęcia dla wpisanego numeru rejestracyjnego\newline

\hspace{0cm}\textbf{Scenariusz główny: }
\begin{enumerate}
\item System wyświetla formularz poprawiania danych
\item Operator poprawia dane: wybiera poprawne zdjęcie lub wprowadza poprawny numer rejestracyjny
\item System przetwarza wprowadzone dane
\item System wyświetla potwierdzenie zmian
\end{enumerate}

\hspace{0cm}\textbf{Scenariusz alternatywny: }
\begin{enumerate}
\item[3.a] System nie może przetworzyć nowych danych
\item[3.a.1] System wyświetla ponownie formularz, zaznaczając, które pola zostały niepoprawnie wypełnione
\item[3.a.2] Następuje powrót do punktu 3 scenariusza głównego
\end{enumerate}
